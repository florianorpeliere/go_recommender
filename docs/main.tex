\documentclass[12pt]{article}

\usepackage[english]{babel}
\usepackage[utf8]{inputenc}
\usepackage{amsmath,amsthm,amssymb}
\usepackage[colorinlistoftodos]{todonotes}

\title{Predicting Ratings with Neighborhood-Based
Methods}

\author{Florian Orpeli\`ere}

\date{\today}

\begin{document}
\maketitle

\section{Formules de bases}
\label{sec:explication}

Posons les \'el\'ements suivants : 

\begin{itemize}
\item La moyenne des notes d'un utilisateur $u$ est $\mu_u$. $I_u$ repr\'esente l'ensemble des items que l'utilsateur $u$ a not\'e :
\newline\newline
$\displaystyle \mu_u = \frac{\sum_{k \in I_u} r_{uk}}{\left| I_u \right|}$
\newline\newline
\item La similarité de Pearson (g\'en\'eral):
\newline\newline
$\displaystyle Sim(x,y)=Pearson(x,y)={\frac {\sum _{i=1}^{n}(x_{i}-{\bar {x}}).(y_{i}-{\bar {y}})}{{\sqrt {\sum _{i=1}^{n}(x_{i}-{\bar {x}})^{2}}}{\sqrt {\sum _{i=1}^{n}(y_{i}-{\bar {y}})^{2}}}}}$
\newline\newline\newline
O\`u ${\bar {x}}={\frac {1}{n}}\sum _{i=1}^{n}x_{i}$ est la moyenne des $x$.
\newline
On pourrait en utiliser d'autres. Celle-ci à la propriété d'avoir des valeurs comprises entre -1 et 1.
\newline\newline
$\displaystyle s_{uj} = r_{uj} - \mu_u$
\newline\newline
O\`u $r_{u_j}$ est la note d'un utilisateur $u$ sur l'item $j$.
\end{itemize}

\section{Neighborhood Models bas\'e sur les utilisateurs}

On commence par calcul\'e la similarit\'e entre deux utilisateurs $u$ et $v$.
\newline\newline
$\displaystyle Sim(u,v)
\newline
=Pearson(u,v)
={\frac {\sum _{k \in{I_u\cap{I_v}}}(r_{uk}-{\mu_u}).(r_{vk}-{\mu_v})}{{\sqrt {\sum _{k \in{I_u\cap{I_v}}}(r_{uk}-{\mu_u})^{2}}}{\sqrt {\sum _{k \in{I_u\cap{I_v}}}(r_{vk}-{\mu_v})^{2}}}}}
\newline
={\frac {\sum _{k \in{I_u\cap{I_v}}} s_{uk}. s_{vk}}{{\sqrt {\sum _{k \in{I_u\cap{I_v}}}s_{uk}^{2}}}{\sqrt {\sum _{k \in{I_u\cap{I_v}}}s_{vk}^{2}}}}}$
\newline\newline
Avec
${\sum _{k \in{I_u\cap{I_v}}}(r_{uk}-{\mu_u}).(r_{vk}-{\mu_v})}$ qui est le produit scalaire des vecteurs $r_{u}$ et $r_{v}$ minor\'e par leur moyenne $\mu_u$ et $\mu_v$ respectivement. Donc lle produit scalaire des notes d'un utilisateurs $u$ et $v$ centr\'ees sur la moyenne.
\newline\newline
Puis, $\sqrt {\sum _{k \in{I_u\cap{I_v}}}(r_{v_u}-{\mu_u})^{2}}$ et $\sqrt {\sum _{k \in{I_u\cap{I_v}}}(r_{v_k}-{\mu_v})^{2}}$ qui est la norme des vecteurs $r_{u}$ et $r_{v}$ minor\'e par leur moyenne $\mu_u$ et $\mu_v$.
\newline\newline
Nous pouvons pr\'edire la note d'un utilisateur $u$ sur l'item $j$ :
 \newline\newline
$\displaystyle r_{uj}= \mu_u + {\frac {\sum _{v \in{P_u(j)}} Sim(u,v) . s_{vj}}{\sum _{v \in{P_u(j)}} |Sim(u,v)|}}$
\newline\newline

\end{document}

